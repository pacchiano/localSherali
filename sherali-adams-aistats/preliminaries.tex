\cite{rowland2017conditions}. We study graphical models over $n$ variables $\{x_1, \cdots, x_n\}$ where $x_i \in \mathcal{A}$. We denote the vector $(x_1, \cdots, x_n)$ as  $x_{[n]}$ or simply as $x$. For any subset $S \subset [n]$ we use the notation $x_S$ to denote the image of $x_{[n]}$ when projected to the subset of variables indexed by $S$. Let $w(S, x_S)$ be  called a "weight" vector indexed by subsets of $[n]$ and assignment projections of $x_{[n]}$ onto $S$. For probabilisitc graphical models where the energy function can be written as $\exp( f(x_{[n]}))$ where $f \mathcal{A}^n \rightarrow \mathbb{R}$ is defined as:

\begin{equation}\label{eq_map}
f(x) = \sum_{S, x_s} w(S, x_s)
\end{equation}

The probability distribution is specified by $p(x) = \exp(f(x))$. The MAP inference problem for $p$ is equivalent to finding $x_* = \arg\max_{x \in \mathbb{X}^n} f(x)$. It can be shown that this problem is NP hard in general. Multiple approaches to get around this problem have been proposed. Most notably, to solve instead an LP relaxation of \ref{eq_map}. 

The Shearli Adams hierarchy of LP relaxations and polytopes $\{\mathbb{L}_k\}_{k=2}^n$ provides a progressively stricter relaxation  such that $\mathbb{L}_n \subset \cdots \mathbb{L}_2$ and solving the objective function over $\mathbb{L}_n$, also known as the marginal polytope $\mathbb{M}$ yields the right answer for \ref{eq_map}. We say that an LP relaxation is tight for a weight vector $w(S, x_S)$ if it has the same value as solving for $\max_{x} f(x)$. We will mostly restrict ourselves to binary graphical models where $\mathcal{A} = \{0,1\}$. We work with a slightly more general version of the Sherali Adams hierarchy. 



\begin{definition}[Marginal Polytope]
We define the marginal polytope $M(\mathcal{X})$ over binary $\{0,1\}$ variables $\mathcal{X}$ with $|\mathcal{X}| = n$ as $M (\mathcal{X}) \subset \mathbb{R}^{3^n}$ such that if $\mu \in M(\mathcal{X})$ then the dimensions of $\mu$ are indexed by $(S, x )$ with $S \subset \{1, \cdots, n\}$ ($S$ is a subset of the variables $\mathcal{X}$) and $x \in \{0,1\}^{|S|}$ is an assignment for these variables and the following relationship must be satisfied:
\begin{equation}
\mu(S, x) = \mathbb{P}_\mu( S = x)
\end{equation}

Each $\mu$ encodes (consistently) a probability distribution over $\mathcal{X}$ by specifiying the probabilities of all possible events in the sigma algebra generated by the random variables  $\mathcal{X}$ (pairs $(S, c)$).

The MAP inference problem over a MRF with weight vector $w(S, x_S)$ is equivalent to solving the following LP:

\begin{equation}
\max_{\mu \in \mathbb{M}(\mathcal{X}) } \langle \mu, w \rangle
\end{equation}

\end{definition}

For all $y \in \{0,1\}^n$, define $\mu_y$ as:

\begin{equation}\label{eq_vertex}
\mu_y(S,x) = \begin{cases}
             1 & \text{ if } x = y_S \\
             0 & \text{ o.w. } 
            \end{cases}
\end{equation}

These are the $2^n$ integral vertices of $M(\mathcal{X})$. Their convex combinations generate all of $M(\mathcal{X})$. These correspond to all deterministic distributions over these $n$ binary random variables. 
\newline

\begin{definition}[Nested Set system ] 
Let $\mathcal{Y}$ be a base set, and $\mathcal{P}(\mathcal{Y})$ be its power set. A nested family of sets $\mathcal{S} \subset \mathcal{P}(\mathcal{Y})$ is such that $S' \in \mathcal{S}$ for all $S' \subset S$ such that $S  \in \mathcal{S}$. Notice that $\emptyset \in \mathcal{S}$ for any nested set system. 
\end{definition}


Let $\mathcal{X}$ be a set of variables with $|\mathcal{X}| = n$ and $\mathcal{S}$ a set system over $[n]$. Call $S_r$ to the number of subsets of $\mathcal{S}$ of size $r$. The number of possible $\{0,1\}$ assignments for all subsets in $\mathcal{S}$ is $\mathcal{N}^{(2)}(\mathcal{S}) = \sum_{i =0}^n |S_i|*2^i$. 

If $\mathcal{S}$ is a nested set system, call $\mathcal{S}^{top} = \{ S \in \mathcal{S} | \not\exists S'\in \mathcal{S}, |S'
| > |S| \text{ s.t. } S \subset S'\}$
\newline


\begin{definition}[Generalized Sherali Adams Hierarchy]
Given a nested set system $\mathcal{S}$ over set $[n]$, let $\mathbb{L}_{\mathcal{S}}$ to be the \emph{generalized sherali adams polytope} of vectors $\mu \in \mathbb{R}^{\mathcal{N}^{(2)}(\mathcal{S})}$ defined for $S \in \mathcal{S}$ and $x \in \{0,1\}^{|S|}$:


\begin{equation}\label{objective}
\mu(S, x)  = \mathbb{P}_{S'}(S = x ), \text{  }  \forall S' \in \mathcal{S}^{top}             
\end{equation}

Where $\mathbb{P}_{S'}(S = x )$ stands for the marginal probability of $S = x$ of the distribution indexed by $S'$. This definition implicitly specifies there must be consistency between the marginalizations from any two distinct $S, S'' \in \mathcal{S}^{top}$ down to a common set $S$ such that $S \subset S'$ and $S \subset S''$. 
\end{definition}

Given a base set $\mathcal{X}$, define $\mathbb{B}_k$ to be the nested set system of all sets of size $\leq k$. 
\newline

\begin{definition}[Sherali Adams Hierarchy]
The $k-$th Sherali Adams polytope over set $[n]$ equals $\mathbb{L}_{\mathbb{B}_k}$. We will use the shorthand notation $\mathbb{L}_k$ instead. 
\end{definition}

\begin{definition}[Pointed Sherali Adams Polytope]
Let $x \in [n]$. Define $\mathbb{B}_k^x = \mathbb{B}_k \cup \{ \{x\} \cup s | s \in \mathbb{B}_k \}$. The pointed Sherali Adams Polytope over set $[n]$ with special variable $x$ equals $\mathbb{L}_{\mathbb{B}_k^x}$. We will use the shorthand notation $\mathbb{L}_k^x$ instead.  
\end{definition}

Denote by $W_k = \{w |  w(S, x) = 0 \forall S | |S| > k\}$. We refer to $w(S.x)$ for $|S| = k$ as the weights of degree $k$. 

